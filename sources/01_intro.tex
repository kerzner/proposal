%%% -*-LaTeX-*-

\chapter{Introduction}

This proposed dissertation is about a framework that describes why and how to use creativity workshops in visualization design studies. In its formative work, we conducted two design studies where we used creativity workshops~\cite{Kerzner2017} and creativity methods~\cite{Kerzner2015} to quickly establish rapport with project stakeholders, to characterize the problems faced by analysts, and to explore solutions to those problems. We based our creativity methods and workshop on descriptions in the existing visualization literature, but we had many unanswered questions about how to effectively use them. To investigate these questions, we analyzed our use of creativity workshops and the experiences of 6 visualization researchers who have used creativity workshops in 15 projects. Based on this analysis and an extensive literature review, the primary anticipated contribution of this proposed dissertation is an actionable visualization creativity workshop framework grounded in creativity theory. It will describe theoretical and practical considerations of creativity workshops, including why it is important that they emphasize creativity and how to customize them to a specific project. This framework will provide guidance for designers to use creativity workshops in their own projects, connect that guidance with existing creativity theory, and identify opportunities for future research on creativity workshops in visualization design.

\section{Overview}

As visualization designers, we aim to create useful visualizations for data analysts in design studies~\cite{Sedlmair2012}. But creating useful visualizations is hard for a variety of reasons. Often, analysts cannot tell us what they want to do with visualization tools because humans struggle to communicate tacit knowledge~\cite{Polanyi1974}. This is further complicated by imperfect communication due to the specialized knowledge of both designers and analysts---we may not know their domain's vocabulary and they may not know what is possible with visualization~\cite{Wijk2006}. We also have to navigate the context where analysts work, accounting for the bureaucracy and political environment of their organizations~\cite{Sedlmair2010}. And we have to do all this while remaining productive as analysts expect to see rewards, often in the form of useful visualizations, for committing their limited time and energy to working with us~\cite{Sedlmair2010}. 

Visualization process models describe the steps of conducting a design study and define three categories of common challenges faced by designers: intellectual, organization, and interpersonal~\cite{McKenna2014,Munzner2009,Tory2004a,Lloyd2011,Sedlmair2010,Sedlmair2012}. \textbf{Intellectual challenges} relate to finding real needs for analysts and whether fulfilling those needs is an interesting visualization problem~\cite{Sedlmair2012}, such as eliciting requirements based on analysts' tacit knowledge~\cite{Polanyi1974}. \textbf{Organizational challenges} include political, bureaucratic, and administrative constraints, such as the ability to share sensitive data~\cite{Sedlmair2010} or the buy-in from management~\cite{Kerzner2015}. \textbf{Interpersonal challenges} focus on relationships or communication between individuals~\cite{Sedlmair2012}, for example, the amount of engagement that analysts have in the project~\cite{Sedlmair2010}. Failure to address any of these challenges undermines the project as seemingly small errors cascade through the design process~\cite{Munzner2009}.

The typical methods for navigating these challenges require significant time commitment, especially at the start of design studies. Commonly, designers use a mix of contextual inquiry and interviews for finding interesting domain problems~\cite{Sedlmair2012}. But these methods require the focused attention of analysts, potentially over multiple days~\cite{Holtzblatt1993}. This strains the collaboration as it requires a large time commitment before we can deliver useful visualizations. Large organizations complicate things further as we must work with a diverse set of analysts, requiring even more time and energy from them as we piece together the organization's goals from disparate perspectives~\cite{Kerzner2015}. 

Recently, designers used creativity methods and workshops to understand analysts' needs while accounting for intellectual, organization, and interpersonal challenges~\cite{Goodwin2013}. {\bf Creativity methods} are participatory activities that explicitly stimulate creative thinking \cite{Osborn1953}. {\bf Creativity workshops} are the structured use of one or more creativity methods, typically with about 10 participants in sessions ranging in length from 2 hours to 2 days~\cite{Jones2005}. The characteristics of creativity methods and workshops include repeated cycles of divergent and convergent thinking, interpersonal leveling, and open communication~\cite{Osborn1953}. Applied to design studies, they help designers and analysts explore the relevant problem and solution spaces~\cite{Goodwin2013}, aid in navigating organizational hierarchies~\cite{Kerzner2015}, and establish relationships with all project stakeholders~\cite{Goodwin2016}.

In the formative work of this proposed dissertation, we applied creativity methods and workshops in two design studies. In our first design study, we used creativity methods with defense analysts to make use of limited meeting time and to gain trust and buy-in from analysts' management~\cite{Kerzner2015}. In our second design study, we used a creativity workshop to understand the diverse analysis needs of neuroscientists and to engage all members of an academic laboratory~\cite{Kerzner2017}. 

We based our use of creativity methods and workshops on examples from the visualization literature~\cite{Goodwin2013,Dykes2010,Walker2013,Lloyd2011}. Although this literature reported enough information for us to use the methods, we did not understand how they could be tailored to our specific needs or why they were successful. The research reported on \emph{what} was done in the workshops. It left us to fill in the practical considerations, \emph{how} to use them, and the underlying theory, \emph{why} they were useful. 

This proposed dissertation will contribute a framework to demystify both the process and theory of using creativity workshops in design studies. The framework will be based on our analysis of qualitative data from 15 creativity workshops in a variety domains. It will also be grounded in creativity theory from an extensive literature review. It will answer questions that designers may have about creativity workshops, including: Why should we care about creativity in visualization design? What does creativity mean in this context? What are creativity methods and workshops? How do we decide if a creativity workshop will help our project? How do we plan a creativity workshop? How do we run a creativity workshop? And, how do we use the workshop output during the visualization design process?

Some of these questions have been addressed by the domains of creative problem solving, software engineering, and psychology, yet, this work fails to account for the nuances of visualization design. We will account for these nuances, including the use of specialized process models~\cite{Tory2004a}, the critical role of data early in the design process~\cite{Munzner2009}, the sharing of knowledge between designers and analysts~\cite{Wijk2006}, the fuzzy nature of visualization software requirements~\cite{Sedlmair2012}, the evolution of data and tasks that occurs throughout the project~\cite{McCurdy2016_Action}, and the organizational, intellectual and interpersonal challenges~\cite{Sedlmair2012}.

\section{Contributions}

This proposed dissertation's primary anticipated contribution is a {\bf visualization creativity workshop framework} to provide practical and theoretical guidance on their use in design studies. It will be supported by three main pillars: 1) reviewing literature on creativity workshops spanning various domains, including, psychology~\cite{Csikszentmihalyi1997,Gardiner1993,Kaufman2006,Sawyer2006,Sawyer2003}, problem solving~\cite{Michalko2006,Gordon1961,Couger1993,DeBono1983,Osborn1953,Miller1989}, and software engineering~\cite{Sanders2008,Dove2015,Sherwin2011,Muller1993,Sanders2010,Mahaux2014,Mahaux2014,Mahaux2007,Jones2008,Jones2007,Maiden2007,Maiden2005,Maiden2004}; 2) analyzing qualitative data collected from the experiences of 6 creativity and visualization experts\footnote{If you are curious: Miriah Meyer, Jason Dykes, Sara Jones, Sarah Goodwin, David Rogers (maybe), and Graham Dove (maybe)} who have used creativity methods and workshops in 15 projects~\cite{Dykes2010,Goodwin2013,Goodwin2016,Koh2011,Walker2013,Rogers2016,Jones2008,Jones2007,Nobre2017,Horkoff2015,Lisle2017,Dove2015}; and 3) analyzing our own experience using creativity methods and workshops in design studies~\cite{Kerzner2015,Kerzner2017}.

This proposed dissertation's secondary contributions are design studies where we used creativity methods and workshops. First, in a design study for defense analysts, we used creativity methods to make use of limited meeting time with our collaborators. This project's contributions include task analysis, data abstraction, and a visualization tool for analyzing spatial and non-spatial ballistic simulation data~\cite{Kerzner2015,Gribble2014}. Second, in a design study for neuroscientists, we used a full-day creativity workshop to expose shared needs and establish rapport with analysts. This project's contributions include a set of software requirements for multivariate graph analysis and two techniques for visualizing graph connectivity~\cite{Kerzner2017,Lauritzen2016}.

The secondary contributions provide important grounding for the proposed creativity workshop framework. Conversely, the creativity workshop framework synthesizes our experience with that of our collaborators and an extensive creativity literature review to propose best practices for using creativity workshops in design studies.

\section{Structure of this Proposal}

The remainder of this proposal includes related work, followed by a discussion of our three projects organized by time. Chapter~\ref{ch:related} contains work related to visualization creativity workshops. Chapter~\ref{ch:vulnerability} reports the design study with defense analysts. Chapter~\ref{ch:connectome} summarizes the design study with neuroscientists. Chapter~\ref{ch:creativity} outlines the proposed remaining work, the visualization creativity workshop framework. Finally, chapter~\ref{ch:conclusion} proposes a timeline for the remaining work. 


%The typical methods to navigate the organizational and intellectual challenges require significant time commitment, especially when working in large organizations. For example, contextual inquiry~\cite{Holtzblatt1993} requires the undivided attention of analysts and designers. 

%is generally recognized  increases linearly with the number of analysts in the organization. This also puts a burden on designers, as we try to piece together a big picture of domain goals based on analysts' disparate perspectives~\cite{Kerzner2015}.

%We also must establish rapport and trust with analysts while navigating the bureaucracy of their organization. 

%Typical methods to navigate these challenges require significant time commitment. We commonly rely on interviews and contextual inquiry to understand analysts' needs, but contextual inquiry, for instance, must be spread over days or weeks~\cite{Holtzblatt1993,Sedlmair2012}.  

%More specifically, in our experience conducting three design studies~\cite{Kerzner2015,Kerzner2016,Kerzner2017}, we were challenged early on to understand analysts' needs from their tacit knowledge, to find shared needs from diverse analysts, to establish rapport with analysts, to navigate the bureaucracy around data analysis, and to do so while under pressure to remain productive as analysts expected to see useful visualizations early in our projects.





%To create useful visualizations, we must understand the analysts' domain problem, relevant data, and analysis tasks~\cite{Munzner2009}. Based on that understanding, we create prototype visualizations. These prototypes start a design cycle, as analysts give feedback on prototypes that we use to better understand their needs~\cite{Tory2004}.

%Process models describe this cycle and its many nuances for visualization~\cite{Sedlmair2012,Munzner2009,Meyer2012,Koh2011,McKenna2014,Tory2004}. These models, however, often fail to help us start the design cycle. 

%In design studies, we aim to create visualizations for data analysts to solve problems~\cite{Sedlmair2012}. To do this, designers must navigate three types of challenges. grouped them into roughly three categories~\cite{Sedlmair2012}. First, intellectual challenges involve understanding analysts' needs and evaluating whether those needs are interesting visualization problems~\cite{Tory2004}. This often involves translating expert's specialized needs into visualization requirements~\cite{Wijk2006}. Second, organization challenges relate to navigating political, bureaucratic, and administrative constraints~\cite{Sedlmair2010}. Organizational challenges include working with sensitive data~\cite{Sedlmair2010} or establishing trust with analysts' management~\cite{Kerzner2015}. And, third, interpersonal challenges focus on relationships or communication between individuals~\cite{Sedlmair2012}, such as, for example, the amount of engagement that analysts have in the project~\cite{Sedlmair2010}. If we fail to address any of these challenges early on, it can cripple the project as seemingly small errors propagate through the design study~\cite{Munzner2009,Sedlmair2012}.

%Visualization process models guide us through design studies~\cite{Tory2004,Munzner2009,McKenna2014,Meyer2013}. These models describe cycles where we understand a domain problem, propose visualization solutions to that problem, and evaluate visualizations to better understand the problem~\cite{Tory2004}. But these models often fail to help us navigate the interpersonal, intellectual, and organizational challenges at the start of the design cycle. In our experience conducting two design studies~\cite{Kerzner2015,Kerzner2017}, we were challenged early on to understand analysts’ needs from their tacit knowledge, to find shared needs from diverse analysts, to establish rapport with analysts, to navigate the bureaucracy around data analysis, and to do so while under pressure to remain productive as analysts expected to see useful visualizations early in our projects.

%We navigated these challenges with creativity methods and workshops.  Creativity methods promote, among other things, interpersonal leveling, open communication, and repeated cycles of divergent and convergent thinking~\cite{Osborn1953}. Creativity workshops are the structured use of one or more creativity methods, typically with about ten participants in sessions of two hours to two days~\cite{Jones2005}. In our first design study, we used creativity methods with defense analysts to gain trust and buy-in from analysts and their managers~\cite{Kerzner2015}. In our second design study, we used a structured creativity workshop to understand the diverse needs of neuroscientists~\cite{Kerzner2017}. 


%In both of these projects, we used methods and workshops described in visualization literature, but we had to search 


%but we had many open questions about why they creativity worked, and how to make them work better.  


%Although they were successful, we had many unanswered questions spanning both the theoretical, why fostering creativity helped the design process, and practical, how to most effectively use creativity methods. After these projects, we began a collaboration with fellow visualization designers to share best practices for the use of creativity in visualization design. Based on our collective success, we believe that the visualization community would benefit from practical and theoretical guidance on how to use creativity workshops to start the design process by understanding user needs and building rapport with analysts. 

%Typical methods to navigate these challenges require significant time commitment. We commonly rely on interviews and contextual inquiry to understand analysts' needs, but contextual inquiry, for instance, must be spread over days or weeks~\cite{Holtzblatt1993,Sedlmair2012}. This is exacerbated when working with analysts in large organizations as we must piece together various analyst's perspectives to understand the overall domain goals~\cite{Kerzner2015,Sedlmair2010}. We also must establish rapport and trust with analysts while navigating the bureaucracy of their organization. If we fail to address any of these challenges early on, it can cripple the project as seemingly small errors propagate through the design process~\cite{Munzner2009,Sedlmair2012}.

% Generalize challenges
%More generally, the challenges we face early in the design process fit into three categories: intellectual, organization, and interpersonal. Intellectual challenges relate to finding real needs for analysts and whether fulfilling those needs is an interesting visualization problem~\cite{Sedlmair2012}, such as eliciting requirements based on analysts' tacit knowledge~\cite{Holtzblatt1993} and identifying shared needs from diverse analysts within an organization~\cite{Sedlmair2010}.  Organizational challenges include . 

%Methods that foster \emph{group creativity}---the collective creativity that emerges from the interaction and communication between group members~\cite{Sawyer2003}---

% Address challenges
%Creativity methods are useful for starting the visualization design process and navigating the intellectual, organization, and interpersonal challenges as seen by their successful use in a variety of projects~\cite{Kerzner2015,Kerzner2017,Goodwin2013,Goodwin2016,McKenna2014}. In our experience, they establish trust between analysts and designers while explicating the analysis needs of a domain: ``\emph{the structured format helped us to keep on topic and to use the short time wisely}" said an analyst we worked with ``\emph{it also helped us rapidly focus on what were the most critical needs going forward}.''


% Motivate
%This proposed dissertation is motivated by our success using creativity methods at the start of design studies with defense analysts~\cite{Kerzner2015} and neuroscientists~\cite{Kerzner2017}. 

% what will the framework do?
%This proposed dissertation will provide guidance on the practical and theoretical aspects of visualization creativity methods and their execution in creativity workshops. We will answer questions that designers may have about this topic, including: Why should we care about creativity in visualization design? What does creativity mean in this context? What are creativity methods and workshops? How do we decide if a creativity workshop will help my project? How do we plan a creativity workshop? How do we run a creativity workshop? And, how do we use the workshop output during the visualization design process? 

%Some of these questions have been addressed by the domains of creative problem solving, software engineering, and psychology, yet, work from these domains fail to account for the nuances of visualization design. We will account for these nuances, including the use of specialized process models~\cite{McKenna2014,Tory2004}, the critical role of data early on in the design process~\cite{Munzner2009,Lloyd2011}, the sharing of knowledge between designers and analysts~\cite{McCurdy2016,Wijk2006}, the fuzzy nature of visualization software requirements~\cite{Sedlmair2012}, and the evolution of data and tasks that occurs throughout the project~\cite{Sedlmair2012,McCurdy2016}.
