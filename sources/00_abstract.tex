%%% -*-LaTeX-*-
%%% This is the abstract for the thesis.
%%% It is included in the top-level LaTeX file with
%%%
%%%    \preface    {abstract} {Abstract}
%%%
%%% The first argument is the basename of this file, and the
%%% second is the title for this page, which is thus not
%%% included here.
%%%
%%% The text of this file should be about 350 words or less.


The early stages of design studies are challenging as designers must quickly establish an understanding of a domain's needs while fostering engagement with their collaborators. In recent years, designers have used creativity workshops to establish such understanding while generating buy-in from collaborators, but there are currently no actionable guidelines for visualization designers who wish to use creativity workshops in their own projects. The key aim of this proposed dissertation is to contribute an actionable framework and practical recommendations for using creativity workshops in visualization design studies. This contribution will provide guidance for the entire cycle of using creativity workshops in design studies, including, deciding whether a creativity workshop would be useful to a design study; planning the workshop methods, participants, and logistics; running the workshop; analyzing the workshop output; integrating that output into the visualization design process; and reflecting on workshop efficacy for continued development of best practices. It is based on the collective experience of seven visualization researchers who have used creativity workshops, and similar methods, in more than ten design studies.  