%%% -*-LaTeX-*-

\chapter{Background and Related Work}
\label{ch:related}
This proposed dissertation is about a framework that describes how to use creativity workshops in visualization design studies. This chapter focuses on work related to creativity in this context. It summarizes theoretical creativity research, including definitions, models, and research methods. Next, it describes the origin and evolution of creativity workshops from problem solving to software engineering and visualization.

\section{Creativity Research}

Creativity research spans many domains, including philosophy~\cite{Poincare1982,Boden2004,Hadamard1945}, psychology~\cite{Sawyer2006,Plucker1999,Kaufman2006}, the arts~\cite{Dewey1934,Csikszentmihalyi1997,Sawyer2003}, business~\cite{Osborn1953,DeBono1983,VonOech1986}, and computer science~\cite{Shneiderman2005,Hewett2005,Boden2004}. Across these domains, researchers agree that {\bf creativity} is generating ideas that are new and useful, where both new and useful can be defined relative to an individual, to a group of people, or to all of humanity~\cite{Mayer1999}. 

The creativity in visualization design studies involves all of the project stakeholders working together to create a new and useful understanding of a domain problem, to propose new and useful visualization solutions for that problem, and to perform new and useful evaluations of those solutions~\cite{Sedlmair2012}. 

Both individuals and groups can be creative~\cite{Sawyer2003}. {\bf Individual creativity} refers to a single creator~\cite{Sawyer2006}. Although individual creators often communicate with others, this communication may be used selectively~\cite{Sawyer2006}. In contrast, {\bf group creativity} emerges from direct interactions between group members~\cite{Sawyer2003}. It is synergistic as the group creates ideas beyond what its members could do on their own. Individual and group creativity are complementary as group creativity often relies on the creativity of its individual members.

We are interested in group creativity as visualization design involves the project stakeholders creating ideas beyond what they could do on their own~\cite{Goodwin2013}. In other words, creativity workshops are about harnessing group creativity for visualization design.

There is extensive research on individual and group creativity spanning many domains. Next, we summarize common creativity models and research approaches that may be relevant to visualization creativity workshops. For a more complete overview of creativity research please refer to summaries of this field, including works by Sawyer~\cite{Sawyer2006,Sawyer2003},  Sternberg~\cite{Sernberg1999}, and Shneiderman~\cite{Shneiderman2005}.

\subsection*{Creativity Models}

Creativity models provide a useful vocabulary, yet they oversimplify human thought \cite{Sawyer2006}. The {\bf four stage model} describes a linear process of creativity~\cite{Hadamard1945,Poincare1982,Wallas1926}, though in reality it is more likely a cyclical and iterative process~\cite{Sawyer2006}. First, creators \emph{prepare}, learning domain concepts. Second, they \emph{incubate} ideas, forming new mental associations. Third, they have an \emph{insight}, generating a new idea. Finally, they \emph{verify} the idea, expressing and elaborating it. 

By expressing an idea, creators start a feedback loop where their actions change the environment, and this, in turn, changes their idea. For example, an artist's idea for a painting evolves as they see the impact of their paint on a canvas~\cite{Glaveanu2013}. The {\bf action theory of creativity} emphasizes the importance of this feedback loop~\cite{Sawyer2006}. 

The four stage model and action theory emphasize that creativity involves both \emph{thinking} and \emph{doing}, often through many interconnected cycles of ideation and expression. These models straddle philosophy and cognitive psychology as they distill human thought and interaction into simplified processes~\cite{Sawyer2006}. They are supported by various creativity research methods, described next. 

\subsection*{Research Methods}

The research methods most relevant to this proposed dissertation study creativity in the context where it occurs --- the real world, outside of controlled laboratory settings~\cite{Mayer1999}. Although these {\bf contextual creativity} approaches are criticized because they lack experimental control, they are ecologically valid and supported by a broad base of qualitative and quantitative data~\cite{Mayer1999}.

There are three broad types of contextual creativity research. Biographical researchers study creative individuals, often with interviews and observations~\cite{Csikszentmihalyi1997}.  Organizational researchers study creativity in large groups, such as businesses~\cite{Drucker1988}. This includes quantitative surveys to identify shared traits of creative organizations~\cite{Isaksen2000} and analyzing how leaders can foster creativity in their organization~\cite{Amabile2008}. Creativity support researchers synthesize the results of biographical, organizational, and other research, to understand how software can foster creativity, and make recommendations for software designers~\cite{Shneiderman2005}.

There are many other approaches to studying creativity that are not necessarily relevant to this proposed dissertation. The quantitative study of creativity --- such as measuring a group's creative output in a laboratory --- often lacks ecological validity and must be synthesized with contextual approaches to generate useful insight~\cite{Sawyer2006}. Additionally, biological~\cite{Martindale1999} and individual personalty methods~\cite{Plucker1999} attempt to explain creativity as an inherit feature of individuals, but have generally been unsuccessful~\cite{Sawyer2006}. Historical~\cite{Albert1999} and cultural creativity research ~\cite{Lubart1999} explains the change of our socio-cultural perspective of creativity over time. They show that the notion of creativity is a relatively new concept, having evolved in the mid-twentieth century.

From all these approaches to studying creativity, there is a general consensus on how creativity occurs~\cite{Mayer1999}. More specifically, creativity results from hard work, open communication, and many series of small but interconnected insights~\cite{Sawyer2006}. The best practices for fostering creativity, however, have changed over time. This can be seen in the evolution of applied creativity research, spanning the domains of problem solving, software engineering, and visualization. 

\section{Creativity for Problem Solving}

The idea of harnessing creativity methods for problem solving started in the fields of business, advertising, and engineering~\cite{Osborn1953}. These methods are based largely on analyzing personal experience, a form of contextual creativity research.

Explicitly using creativity to solve problems was first proposed by Osborn~\cite{Osborn1953}, where he coined the term \emph{brainstorming} with a militaristic definition of \emph{``using the \emph{brain} to \emph{storm} a creative problem - and to do so in a commando fashion, with each stormer audaciously attacking the same objective."} Effective brainstorming is the freewheeling generation of as many ideas as possible, followed by evaluation to identify the most promising ones. The Creative Problem Solving Group formalized these principles into a repeatable process that is used by consulting firms today~\cite{Isaksen2000}.

Related to brainstorming is Synectics, which encourages problem solving through structured workshops consisting of three phases~\cite{Gordon1961}. First, workshop participants explore a given problem, often in terms of needs or wishes. Next, participant use the ideas as \emph{springboards} for in-depth discussion and elaboration. Third, participants evaluate the ideas and execute the more promising ones. The Synectics methodology has been adopted by various consulting and training companies~\cite{Nolan2003}.

Many consultants have also written books about their experience using creativity for problem solving, often in business settings. de Bono~\cite{DeBono1983} emphasizes ideation through suspending judgment and restructuring ideas. Similarly, von Oech~\cite{VonOech1986,VonOech1988} encourages creators to gain new perspectives on their problems with structured methods. Miller~\cite{Miller1989} proposes that creators should balance analytic methods, such as listing assumptions, with intuitive methods,such as brainstorming.

Many aspects of creativity for problem solving are still in use today, such as suspending judgment~\cite{Osborn1953}, encouraging the use of metaphors and analogies~\cite{Gordon1961}, and promoting reflection~\cite{DeBono1983}. However, on their own these methods are not immediately applicable to visualization design. They assume the workshop participants have sufficient knowledge to reach a valid solution whereas visualization designers recognize that they must establish shared knowledge of visualization and their collaborator's domain for a successful project~\cite{Wijk2006}. 

\section{Creativity Workshops in Software Engineering}

Software requirements engineers translate the specialized knowledge of project stakeholders into concrete requirements. This is a creative process as the engineers must create requirements that define useful software~\cite{Robertson2002}. In some projects, engineers use creativity workshops to elicit needs from project stakeholders~\cite{Jones2005}. These workshops range in duration from 0.5 to 2 days, involve 8 to 24 participants, and typically output hundreds of ideas about software requirements ~\cite{Jones2007}.

The creative problem solving literature provides a foundation for creativity workshops in this domain.  Maiden et al.~\cite{Maiden2004} reported the first creativity workshop in software engineering. It was a two day workshop to create requirements for aircraft scheduling software, using methods to encourage analogical thinking --- similar to Synectics~\cite{Gordon1961}. Another two day workshop run by Maiden et al.~\cite{Maiden2007} used repeated cycles of divergent and convergent thinking --- similar to Creative Problem Solving~\cite{Osborn1953} --- to understand requirements for airspace management software. Similarly, Jones et al.~\cite{Jones2007} used a creativity workshop to create requirements for e-learning software also following a diverge-converge pattern.

Creativity research in software engineering also examines the theory behind workshops. Mahaux et al.~\cite{Mahaux2013,Mahaux2014} describe and validate a set of factors that influence creativity in groups of software engineers, but this work falls short of prescribing best practices for fostering creativity. Recently, Maiden et al.~\cite{Maiden2010} mapped the use of creativity requirements workshops to those of creative problem solving to identify areas for future work. 

There has also been work to analyze the creativity methods for requirements engineering. Horkoff et al.~\cite{Horkoff2015} categorize methods by the structure of their output, arguing that more structured output is effective for engineers. Biskjaer et al.~\cite{Biskjaer2017} propose a framework for analyzing creativity methods based on certain attributes, including their structure and their use of divergent or convergent thinking. And Grube et al~\cite{Grube2008} categorize methods based on where they should be used in the design process. While these works provide important classification of creativity methods, they do not address the challenges of assembling creativity methods into a coherent workshop. 

Moreover, research in software engineering does not immediately apply to visualization design. In this domain, creativity workshops are often part of larger requirements engineering methodologies~\cite{Jones2005}. Their outputs are used to generate formal requirements for software developers~\cite{Maiden2005}. In contrast, visualization software requirements are often ill defined~\cite{Sedlmair2012} and evolve throughout the project~\cite{McCurdy2016_Action}.

\section{Creativity Workshops for Visualization}

Visualization is generally seen as a creative problem, where creativity is needed to explore a broad space of ideas before winnowing down to the more promising ones~\cite{Munzner2009,McKenna2014,Sedlmair2012}. The explicit use of creativity methods in visualization design evolved independently but in parallel to their use in software engineering. 

In visualization, creativity methods and workshops were first described as part of design studies. Dykes et al.~\cite{Dykes2010} used a full day workshop to understand the visualization needs of geographic information systems analysts. Walker et al.~\cite{Walker2013} used a similar workshop to quickly understand the needs of defense analysts. Kerzner et al.~\cite{Kerzner2015} used creativity methods to engage analysts and their managers in a large organization. These projects did not describe their work as creativity methods or workshops, but in retrospect these are early examples of explicitly fostering creativity for visualization design.

Goodwin et al.~\cite{Goodwin2013} introduced the term \emph{creativity workshop} to visualization. They customized software engineering creativity workshops for a design study with energy analysts. Kerzner et al.~\cite{Kerzner2017} applied the same workshop structure to understand the needs of neuroscientists. Recently, Goodwin et al.~\cite{Goodwin2016} again used a similar workshop structure to understand the needs of constraint programmers.

Different creativity workshop structures have also been used in visualization research. Nobre et al~\cite{Nobre2017} used a two hour workshop to understand how genealogists could use their visualization software. Lisle et al.~\cite{Lisle2017} used a two day creativity workshop to find opportunities for a collaboration between visualization designers and evolutionary biologists. Rogers et al.~\cite{Rogers2016} brought together visualization designers to explore potential solutions to specific problems in the domains of civil engineering and oceanography.

Despite these repeated success, understanding how to use creativity workshops requires that designers must piece together disparate information from the literature of visualization, software engineering, problem solving, and psychology. There are no guidelines nor best practices for applying creativity workshops to visualization design. This proposed dissertation will identify such best practices, based on the analysis of \emph{every} visualization project mentioned in this section. Before proposing the framework, we present our completed formative work that used creativity methods and workshops in two design studies.