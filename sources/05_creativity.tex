%%% -*-LaTeX-*-

\chapter{Proposed Work: Visualization Creativity Workshops}
\label{ch:creativity}

The previous two chapters described design studies where we used creativity methods and workshops ad hoc. We ran them based on descriptions in the literature --- in retrospect, we did not understand why creativity is important, how to tailor creativity methods to fulfill our goals, nor the best practices for encouraging creativity. This is because there are currently no established guidelines or frameworks that describe how to use visualization creativity workshops. To explain the theory behind creativity workshops and to provide practical recommendations based on our experiences, we propose creating an actionable \emph{visualization creativity workshop framework} grounded in theory. 

In this chapter, we describe the research methods and data that we will use to create the framework. We outline the framework with set of 36 open questions. And we describe the project's remaining work. The anticipated outcome is a publication in IEEE Transactions on Visualization and Computer Graphics (IEEE TVCG).

\section{Data and Methods}

The proposed framework will be based in the analysis of creativity methods and workshops in a variety of projects. We are working with six creativity and visualization researchers who have also used creativity methods and workshops. Collectively with our collaborators, we have run more than 15 creativity workshops in a variety of domains and settings~\cite{Dykes2010,Goodwin2013,Goodwin2016,Koh2011,Walker2013,Rogers2016,Jones2008,Jones2007,Nobre2017,Horkoff2015,Lisle2017,Dove2015,Kerzner2017,Kerzner2015,Maiden2004,Maiden2005}. We gathered data from a subset of these workshops including the materials, agendas, artifacts, output, and impact on the project's final result. Analyzing these data will provide a practical foundation for our proposed framework.

The framework will also be grounded in creativity theory. We have reviewed literature related to creativity workshops in the fields of psychology~\cite{Csikszentmihalyi1997,Gardiner1993,Kaufman2006,Sawyer2006,Sawyer2003}, problem solving~\cite{Michalko2006,Gordon1961,Couger1993,DeBono1983,Osborn1953,Miller1989}, and software engineering~\cite{Sanders2008,Dove2015,Sherwin2011,Muller1993,Sanders2010,Mahaux2014,Mahaux2014,Mahaux2007,Jones2008,Jones2007,Maiden2007,Maiden2005,Maiden2004}.

We will create the framework through the qualitative analysis of data gathered from our experiences and our literature review. We will apply thematic analysis to identify overarching themes in our data~\cite{Braun2006}. But we also have tacit knowledge about creativity workshops that is not in the data. To capture this knowledge, we will use reflection, analyzing our experiences and proposing new insights based on that analysis~\cite{Boud1985}. More specifically, we will use collaborative reflection, analyzing many individuals' collective experience and articulating the insights of this analysis for future reference~\cite{Prilla2012}. We may also use qualitative analysis methods including open coding and grounded theory~\cite{Corbin1990}.

Through this analysis, we will create a framework that describes how and why to use creativity workshops in visualization design. As we have already started this analysis, we outline the framework in the next section.

%It is based on our experience and on existing theory. Over the past one and a half years, we carefully analyzed our experience with three workshops in more than one hundred hours of conversation~\cite{Goodwin2013,Goodwin2016,Kerzner2017}. And we reviewed literature related to creativity workshops in the fields of psychology~\cite{Csikszentmihalyi1997,Gardiner1993,Kaufman2006,Sawyer2006,Sawyer2003}, problem solving~\cite{Michalko2006,Gordon1961,Couger1993,DeBono1983,Osborn1953,Miller1989}, and software engineering~\cite{Sanders2008,Dove2015,Sherwin2011,Muller1993,Sanders2010,Mahaux2014,Mahaux2014,Mahaux2007,Jones2008,Jones2007,Maiden2007,Maiden2005,Maiden2004}. Through a form of thematic analysis~\cite{Braun2006}, including countless iterations on themes based on conversations with our collaborators, we synthesized the result of those conversations and literature review into a framework outline. 

%This outline will guide the creation of our complete framework as we work with collaborators to analyze our collective experience, including more than 12 creativity workshops a variety of domains~\cite{Dykes2010,Goodwin2013,Goodwin2016,Koh2011,Walker2013,Rogers2016,Jones2008,Jones2007,Nobre2017,Horkoff2015,Lisle2017,Dove2015}. We will use \emph{reflection}, analyzing our experience and proposing new insights based on that analysis~\cite{Boud1985}. More specifically, we will use \emph{collaborative reflection}, analyzing many individuals' collective experience~\cite{Prilla2012}. Collaborative reflection is different than informal discussions, for example, because it has explicit goals that include creating insights and articulating those insights for future reference~\cite{Prilla2012}. We may also use various qualitative research methods as part of our collaborative reflection, including grounded theory~\cite{Corbin1990} and thematic analysis~\cite{Braun2006}. 

%Collaborative reflection and literature review are accepted research methods for creating theoretical frameworks. Popular visualization process and decision models are based largely on these methods~\cite{Sedlmair2010,Munzner2009,Meyer2013,McKenna2014,Tory2004}. Also, collaborative reflection is a type of contextual creativity research as we analyze our data collected from experiences running creativity workshops in real projects~\cite{Sawyer2006}. Although contextual methods lack precise measurement and experimental control, they preserve ecological validity of observations~\cite{Mayer1999}. This is crucial for applied visualization research, as we aim to create insights that are transferable to future visualization projects~\cite{Sedlmair2012}. In our final framework, however, we will hedge the contributions according to the research methods used. Moreover, we hope that this framework provides appropriate language to enable future research on visualization creativity workshops.

\section{Visualization Creativity Workshop Framework}

The visualization creativity workshop framework consists of six stages that are based roughly on the steps that a designer would use in planning, running, and evaluating a workshop. The stages are: 1) {\bf motivate} the use of creativity workshops; 2) {\bf scope} the workshop focus and goals; 3) {\bf plan} the workshop methods and logistics; 4) {\bf run} the workshop; 5) {\bf analyze} the workshop output; and 6) {\bf reflect} on the workshop efficacy. For each of these stages, we describe its purpose, input, output, and a list of open questions that will need to be answered in our final framework.

These stages are an abstraction of a complex and messy process. They provide temporal constructs to organization the discussion of our experiences with creativity workshops. The remainder of this section outlines the six stages, followed by some general discussion questions that will need to be addressed in our publication.

\subsection*{Motivate}

The motivate phase is precondition to using creativity workshops. It is meant to convince designers that workshops are useful for design studies. The questions that a designer may have about this include:

\begin{enumerate}
    \item \emph{Why should I use a workshop? (Aren't interviews good enough?)}
    \item \emph{Why should my workshop be structured? (Can I just meet informally with my collaborators?)}
    \item \emph{What exactly does creativity mean in this context?}
    \item \emph{Why is it important that the workshops emphasize creativity?}
    \item \emph{Why do I need a \emph{visualization} creativity workshop framework? (Isn't the existing literature good enough? What are the nuances of visualization design that I need to account for in my workshop?)}
\end{enumerate}

\subsection*{Scope}

In the scope phase, we evaluate whether a workshop would be useful for a design study. Input to this phase is a design study with domain experts. Outputs from it include a workshop focus (what role it will serve in the design process e.g., to understand or to ideate~\cite{McKenna2014}) and goals (the stated reason for running the workshop). A designer may ask:

\begin{enumerate}
    \item \emph{Where is a good point in the design process to run a workshop? (Focus)}
    \item \emph{How much contextual knowledge do I need to run a workshop?}
    \item \emph{What can my collaborators and I expect to get from a workshop? (Goals)}
    \item \emph{Are my project constraints amenable to a workshop?}
\end{enumerate}

\subsection*{Plan}

In the plan phase, we assemble a workshop that fulfills the goals and focus while recruiting contributors. Input to this phase is a workshop scope --- the focus and goals. Output from it are a list of contributors (including facilitators, scribes, and participants), logistics (such as venue and duration), and methods (the activities planned for the workshop). The open questions are:

\begin{enumerate}
    \item \emph{Who should I recruit as contributors?}
    \item \emph{What should I consider in my logistics  --- venue and duration?}
    \item \emph{What are the different kinds of methods available for workshops?} %\ek{This question and the next one will probably be their own section in the final paper. There's a lot of interesting stuff we can talk about here: analytic vs intuitive, structured vs unstructured, paradigm preserving vs paradigm breaking, etc}
    \item \emph{How should I select methods to use in the workshop?}
    \item \emph{How should I consider the four stage model of creativity while planning my workshop?}
    \item \emph{How should I consider the action theory of creativity while planning my workshop?}
    \item \emph{Should I run a pilot workshop?}
\end{enumerate}

\subsection*{Run}

In the run phase, we execute the workshop and collect artifacts from it. Input to this phase is a workshop plan, including the contributors, logistics and methods. Output is a successfully executed workshop along with tangible and intangible results. A designer may ask:

\begin{enumerate}
    \item \emph{How should I prepare workshop contributors? (e.g., surveys of participants) }
    \item \emph{What are best practices for running the workshop?}
    \item \emph{How should I record ideas during the workshop?}
    \item \emph{How should I collect artifacts from the workshop?}
\end{enumerate}

\subsection*{Analyze}

In the analyze phase, we make sense of the tangible and intangible workshop results. Input to this phase are the results of a workshop that has recently been run. Output is actionable knowledge that fulfills the workshop focus and goals. The open questions are:

\begin{enumerate}
    \item \emph{What does the typical workshop output look like?}
    \item \emph{What are the different ways that I can make sense of workshop output?}
    \item \emph{How involved should workshop contributors be in analyzing the output?}
    \item \emph{How can I use workshop output in generative design methods?}
    \item \emph{How can I use workshop output in evaluative design methods?}
\end{enumerate}

\subsection*{Reflect}

In the reflect phase, we evaluate the efficacy of the workshop. Input to this phase are the scope, plan, execution and analysis. Output are insights for the visualization community, potentially transferable to future design studies. The open questions are:

\begin{enumerate}
    \item \emph{How should I collect feedback from workshop contributors?}
    \item \emph{How should I evaluate the workshop with respect to the scope and plan?}
    \item \emph{When and how should I evaluate workshop effectiveness?}    
    \item \emph{What is the role of quantitative/qualitative evaluation methods in our reflection?}
    \item \emph{What should I share about my workshop and reflection with the visualization community?}
\end{enumerate}

\subsection*{Discussion}

There are also interesting questions about creativity workshops that do not fit into the process of using them. These questions will be addressed in the discussion of our proposed work:

\begin{enumerate}
    \item \emph{What are the limitations of using creativity workshops?}
    \item \emph{How effective are creativity workshops with casual or non-expert users?}
    \item \emph{What is the relationship between visualization creativity workshops, participatory design, and co-design?}
    \item \emph{What is the relationship between visualization creativity workshops and agile development?}
    \item \emph{How do visualization creativity workshops apply to other areas of visualization research, such as technique or algorithm-driven work?}
    \item \emph{How can we validate the creativity workshop framework?}
\end{enumerate}

\section{Anticipated Publications and Remaining Work}

The remaining work of this project will expand this outline into a full framework for publication in IEEE TVCG. Writing this framework is an important part of our analysis as we will be forced to articulate our assumptions and ideas~\cite{Richardson2005}. We will also use the writing as a medium for collaborative reflection as we ask our collaborators to either support or refute what we have written. Through this writing, we will explicate our framework and identify best practices (or pitfalls) for applying creativity workshops to visualization design studies. The next chapter describes a more detailed timeline of the remaining work for this publication.